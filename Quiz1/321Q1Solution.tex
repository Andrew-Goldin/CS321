CS 321
Quiz 1
Total Points:
For each one of the first six questions below, circle either "True" or "False" unambiguously. For the last quesiton, you should write a justification in case you need partial credit.

  1. If f(n) = O(g(n)), then g(n) is an asymptotically tight upper bound for f(n). FALSE

  2. 50N^2 + 10N + 99 = \Omega(N^2). TRUE

  3. If f(n) = \Theta(g(n)), then g(n) is asymptotically tight lower bound for f(n). TRUE

  4. If f(n) = \Omega(g(n)), then f(n) = \Theta(g(n)). FALSE

  5. If f(n) = \Theta(g(n)), then f(n) = O(g(n)). TRUE
  
  6. If f(n) = \Theta(g(n)), then g(n) = O(f(n)). TRUE
  
  7. What is the smalles INTEGER value of N_0 that can be used in a proof of [5N^2 = O(N^3)] assuming that c MUST be equal to 2?
  
  N_0 = 3, because 2 does not work, and we were asked for the smallest integer.
  
  5(2^2) <= c * (2^3)
  5(4) <= 2 (8)
  20 <= 16 - THIS IS NOT TRUE, BUT if you replace N with 3, it works.
